%-------------------------------------------------------------------------------
%	SECTION TITLE
%-------------------------------------------------------------------------------
\cvsection{Self-motivated projects}


%-------------------------------------------------------------------------------
%	CONTENT
%-------------------------------------------------------------------------------
\begin{cventries}

%---------------------------------------------------------
%  \cventry
%    {Java, Graph theory, Servlet API}
%    {Maze generator}
%    {\color{awesome-skyblue}\href{https://github.com/zhukovsd/maze-generator}{\underline{GitHub project}}}
%    {2015 - 2016}
%    {
%      \begin{cvitems} % Description(s) of experience/contributions/knowledge
%        \item {\color{awesome-skyblue}{Web application - \href{http://zhukovsd.github.io/maze-generator/}{http://zhukovsd.github.io/maze-generator/}}}
%        \item {Algorithm works with planar graphs, which allows building mazes of arbitrary shape}
%        \item Finished generator was user during the implementation of maze solving Android game
%      \end{cvitems}
%    }

%---------------------------------------------------------
%  \cventry
%    {Java, MongoDB, concurrency, web sockets, HTML5}
%    {Multiplayer minesweeper on infinite grid}
%    {\color{awesome-skyblue}\href{https://github.com/zhukovsd/endless-field}{\underline{GitHub project}}}
%    {2016}
%    {
%      \begin{cvitems} % Description(s) of experience/contributions/knowledge
%        \item {\color{awesome-skyblue}{Web application - \href{http://5.101.123.222:8080/online-minesweeper/}{http://5.101.123.222:8080/online-minesweeper/}}}
%        \item {Backend - JavaEE application, Frontend - HTML5/JS/Canvas, communication protocols - HTTP, WebSocket}
%        \item {The grid is conventionally infinite and lazy-generated. A single unit is a chunk - an area of NxM cells}
%        \item A chunk is a unit of every internal interaction within the application:
%        \begin{itemize}
%			\item Cells fetching from DB happens by indexed chunkID field
%			\item Fine-grained locking mechanism locks the field by chunks
%            \item Field area requesting by clients
%		\end{itemize}
%      \end{cvitems}
%    }
    
%---------------------------------------------------------
%   \cventry
%     {Kotlin, Spring, ES6, PugJS, Webpack, CircleCI, Docker Swarm}
%     {Typing test}
%     {\color{awesome-skyblue}\href{https://github.com/zhukovsd/typing-test}{\underline{GitHub project}}}
%     {2018}
%     {
%       \begin{cvitems} % Description(s) of experience/contributions/knowledge
%         \item {Running instance of the app - {\color{awesome-skyblue}\href{http://88.99.12.213/}{http://88.99.12.213/}}}
%         \item {A simple 1-minute typing speed test with frontend built with ES6/Pug.js. Each test result gets submitted to the backend (built with Kotlin/Spring) which responds with player's placement and stores typing speed characteristics (characters per minute, typos count) to the persistent Redis DB}
%         \item {The app runs two microservices, the first one encapsulates Tomcat, which runs Spring application. The second microservice runs a Redis instance}
%         \item {CircleCI pipeline builds the app (frontend with Webpack and backend with Maven) and deploys it to the remote Swarm cluster}
%         \item Full project overview is available on GitHub - {\color{awesome-skyblue}\href{https://github.com/zhukovsd/typing-test/blob/master/README.md}{README.md}}
%       \end{cvitems}
%     }

%---------------------------------------------------------
  \cventry
    {}
    {Mentoring, course writing, blogging}
    {\color{awesome-skyblue}\href{https://zhukovsd.github.io/java-backend-learning-course/}{\underline{Practical Java course}}}
    {2017 - current}
    {
      \begin{cvitems} % Description(s) of experience/contributions/knowledge
        \item {As a non-commercial side project, I mentor developers (mainly Java, but also Python, PHP), helping them to build their skills and get a job}
        \item {I've created a {\color{awesome-skyblue}\href{https://zhukovsd.github.io/java-backend-learning-course/}{practical Java course}} (in Russian), consisting of 7 pet project, covering all the major Java Backend topics}
        \item {My one-on-one mentoring to the students includes: creating a learning plan, code review, advice on how to improve their skills, and helping them to write a CV and prepare for interviews}
        \item {4 of my students have already got a job, and 2 more are in the process of interviewing}
        \item {I blog about mentoring on {\color{awesome-skyblue}\href{https://t.me/zhukovsd_it_mentor}{Telegram}}, and stream live coding sessions and some of the project reviews on {\color{awesome-skyblue}\href{https://www.youtube.com/@zhukovsd_it_mentor}{YouTube}}}
      \end{cvitems}
    }

\end{cventries}
